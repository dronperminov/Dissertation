\section{Случай нескольких классов}\label{sec:ch1/multiclass_case}

Рассмотренные ранее методы касаются задачи бинарной классификации, когда множество допустимых меток ограничено двумя классами. Однако на практике часто возникает необходимость классификации объектов в более чем два класс~\cite{aly2005survey}. Переход от бинарной к многоклассовой классификации существенно усложняет как построение, так и интерпретацию модели~\cite{lorena2008review}.

Существуют два стандартных подхода к решению многоклассовой задачи на основе бинарных классификаторов: стратегия \textbf{один против всех} (one-vs-rest) и стратегия \textbf{попарной классификации} (one-vs-one). В первом случае для каждого из \(C\) классов обучается отдельный бинарный классификатор, который отделяет данный класс от объединения всех остальных. Во втором случае для каждой из \(\frac{C(C-1)}{2}\) пар классов строится бинарный классификатор, различающий только эти два класса, а итоговое решение принимается, например, по большинству голосов или с использованием процедуры агрегации~\cite{galar2011overview}.

Обе стратегии имеют как теоретические, так и практические недостатки. В стратегии один против всех возникает проблема перекоса, связанная с несбалансированностью классов. При наличии сильно преобладающего класса классификаторы могут склоняться к частому отнесению объекта к этому классу, даже если признаки ближе к другому. Это приводит к смещению аппроксимации и, как следствие, к снижению обоснованности принятого решения. Дополнительные методы борьбы с дисбалансом, такие как дублирование редких классов или уменьшение выборки преобладающих, искажают исходное распределение данных, что затрудняет интерпретацию результатов и может приводить к потере статистической достоверности.

Стратегия попарных классификаторов, напротив, требует построения большого числа моделей, число которых растёт квадратично с числом классов. Кроме того, процедура выбора итогового класса по результатам попарных голосований может быть неоднозначной~\cite{kang2015constructing}: возможны случаи, при которых отсутствует чёткий победитель. При этом каждая отдельная модель опирается на подмножество данных, и совокупный результат может не учитывать общую структуру пространства признаков. В результате возникает риск потери согласованности между частными классификаторами, что негативно сказывается на устойчивости системы в целом.

Таким образом, обобщение бинарной модели на многоклассовую постановку сталкивается с рядом фундаментальных затруднений. Проблемы интерпретируемости, статистической состоятельности и устойчивости принятия решений становятся особенно острыми при наличии несбалансированных классов и сложной структуры признакового пространства. Эти соображения подводят к необходимости переосмысления самой постановки задачи классификации, особенно в ситуациях, когда интерес представляет лишь один или малое число целевых классов, а остальные данные играют вспомогательную роль.
