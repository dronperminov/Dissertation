\section{Случай нескольких классов}

В случае многоклассовой классификации (\(C > 2\)) предлагаемая конструкция унарных классификаторов сохраняет свою применимость и обладает рядом существенных преимуществ по сравнению с классическим подходом, основанным на многоклассовой нейронной сети или на парных классификаторах ``один против одного''. Прежде всего, при использовании унарной схемы для каждого класса \(c = 1, \dots, C\) строится собственный унарный классификатор, обученный различать носитель класса \(c\) от фонового равномерного распределения.

Таким образом, требуется построить \(C\) независимых классификаторов, каждый из которых решает задачу бинарной классификации в формате ``объекты данного класса против фона''. В отличие от схемы ``один против одного'', где количество классификаторов составляет \(\frac{C(C - 1)}{2}\), унарная схема масштабируется линейно по числу классов и не требует сложных стратегий агрегации результатов голосования.
