\section{Преимущества унарной классификации}

Ключевым достоинством унарного подхода является полная устойчивость к проблеме дисбаланса классов. Каждый классификатор обучается только на положительных объектах своего класса и на независимом фоновом множестве, совпадающим по размеру. Таким образом, влияние других, возможно многочисленных, классов исключается на этапе обучения, и несбалансированность исходного обучающего множества не приводит к смещению в сторону более представленных классов.

Кроме того, каждый классификатор формирует свою собственную аппроксимацию апостериорной вероятности \(c_n^{(i)}(x)\), оценивая степень принадлежности точки \(x\) классу \(i\). Совокупность таких значений \((c_n^{(1)}(x), \dots, c_n^{(C)}(x))\) образует векторную оценку, позволяющую как выбрать наиболее вероятный класс (например, по максимуму), так и сформулировать стратегию отказа, если все оценки не превышают заданного порога \(\beta\). Последнее обеспечивает возможность построения отказоустойчивой классификационной системы, способной помечать сомнительные случаи как требующие дополнительного рассмотрения.

Ещё одним немаловажным преимуществом является модульность архитектуры: поскольку все классификаторы независимы, допускается использование различной архитектуры (в том числе различной глубины и сложности) для различных классов. Это даёт возможность адаптировать модель под особенности каждого из классов, делая систему более гибкой и устойчивой к неоднородности обучающих данных.
