\section{Модель кластеров уровня плотности}

Одним из фундаментальных подходов к выявлению скрытой структуры в данных является использование модели кластеров уровня плотности. Этот подход берёт своё начало в работах \cite{bock1974automatische} и \cite{hartigan1975clustering}, в которых было предложено рассматривать кластеры как области пространства признаков, характеризующиеся повышенной плотностью вероятностного распределения. Интуитивно предполагается, что наблюдения, принадлежащие к одному кластеру, с большей вероятностью сосредоточены в определённой области пространства, в то время как между кластерами плотность стремится к более низким значениям.

В дальнейшем, развитие этой идеи получило формальное обоснование в рамках непараметрических методов оценивания плотности. В частности, в работах~\cite{devroye1980strong} и \cite{devroye1977strong} были представлены строгие доказательства состоятельности таких оценок, как гистограмма, оценка по методу \(k\) ближайших соседей, а также ядерная оценка. Эти результаты впоследствии были систематизированы и обобщены в их монографии по непараметрической статистике, ставшей классической.

На основе этих теоретических результатов в~\cite{wong1983kth} предложили алгоритм кластеризации, основанный на выделении областей высокой плотности, который был реализован в рамках иерархического подхода. Этот метод, получивший название вероятностного метода, позволял строить деревья кластеров на основе последовательного объединения областей с близкими характеристиками плотности. К сожалению, из-за высокой вычислительной сложности, связанной с необходимостью многократного расчёта расстояний между точками и хранения оценок плотности в памяти, его практическое применение оказалось ограниченным малыми объёмами выборок и невысокой размерностью пространства признаков.

Формально, пусть \(f(X)\) -- плотность распределения случайного вектора \(X \in \mathbb{R}^d\). Для любого порогового значения \(c > 0\) вводится \textbf{множество уровня плотности}:
\[
B(c) = \{X \in \mathbb{R}^d : f(X) > c\}.
\]

Это множество представляет собой объединение всех точек, в которых плотность превышает заданное значение \(c\). Модель кластеров уровня плотности предполагает, что каждый кластер соответствует одной связной компоненте множества \(B(c)\):
\[
B(c) = \bigcup_{i=1}^M B_i(c),
\]
где \(B_1(c), B_2(c), \dots, B_M(c)\) -- непересекающиеся связные компоненты, каждая из которых интерпретируется как отдельный кластер.

Данный подход позволяет не задавать количество кластеров заранее, поскольку число связных компонент может изменяться в зависимости от значения \(c\). При больших значениях \(c\) в множестве \(B(c)\) остаются лишь точки, расположенные в наиболее плотных участках пространства, а при уменьшении \(c\) связные области начинают расширяться и объединяться, формируя иерархическую структуру кластеров.

На практике множество \(B(c)\) и его компоненты \(B_i(c)\) недоступны напрямую, поскольку истинная плотность \(f(X)\) неизвестна. Однако с использованием непараметрических оценок плотности можно построить приближённое множество уровня и использовать его для выявления кластеров. Это позволяет задать концептуальную основу для методов обучения без учителя, в которых наличие плотностной структуры в данных служит основой для группировки наблюдений.

Таким образом, модель кластеров уровня плотности обеспечивает строгую вероятностную интерпретацию кластеризации как задачи геометрического разделения множества высокого уровня плотности. Это становится особенно важным в ситуациях, когда отсутствуют априорные сведения о принадлежности объектов к классам, а само разделение должно быть основано исключительно на свойствах распределения наблюдаемых данных.
