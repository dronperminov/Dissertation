\chapter*{Словарь терминов}             % Заголовок
\addcontentsline{toc}{chapter}{Словарь терминов}  % Добавляем его в оглавление

\textbf{Классификация} -- задача машинного обучения, в которой требуется отнести объект к одному из заранее определённых классов.

\textbf{Унарная классификация} -- подход, при котором обучение производится только на объектах одного (целевого) класса, а остальные данные считаются фоновыми или неизвестными.

\textbf{Отказ от классификации} -- механизм, позволяющий модели не принимать решение о принадлежности к какому-либо классу при низком уровне уверенности.

\textbf{Порог доверия \(\beta\)} -- значение, определяющее минимальный уровень уверенности модели, при котором принимается решение о классификации объекта.

\textbf{Многослойный персептрон} -- класс искусственных нейронных сетей, состоящий из нескольких слоёв нейронов, каждый из которых связан с предыдущим полносвязным образом. Используется для аппроксимации сложных функций.

\textbf{Полносвязный слой} -- слой нейронной сети, в котором каждый нейрон соединён со всеми выходами предыдущего слоя.

\textbf{Аппроксимация} -- приближённое представление функции, заданной неявно, с помощью некоторой модели, например нейросети или гистограммы.

\textbf{Градиентный спуск} -- метод оптимизации, основанный на итеративном обновлении параметров модели в направлении антиградиента функции потерь.

\textbf{Оптимизаторы градиентного спуска} -- алгоритмы, используемые для настройки параметров модели в процессе обучения.

\textbf{Синтетические данные} -- искусственно сгенерированные данные, используемые для проверки гипотез, обучения моделей и проведения контролируемых экспериментов при отсутствии достаточного количества реальных данных.

\textbf{Объясняющее дерево} -- структура, позволяющая в интерпретируемой форме представить поведение модели путём построения дерева решений на выходных значениях модели.

\textbf{Нормализация признаков} -- преобразование признаков, приводящее их к единому масштабу для повышения устойчивости и скорости обучения.
