\section{Общая характеристика интеллектуальной системы машинного обучения}

В рамках выполненного исследования была разработана интеллектуальная система машинного обучения, предназначенная для визуального и экспериментального изучения поведения моделей классификации в условиях ограниченного объёма обучающих данных, дисбаланса классов, а также в присутствии фона и пропущенных значений. Система представляет собой автономное клиентское приложение, реализованное на языке JavaScript~\cite{flanagan2013javascript}, не требующее установки, интернет-соединения или использования графического ускорителя, что обеспечивает его широкую доступность и воспроизводимость экспериментов.

Интеллектуальная система предназначена для комплексной демонстрации, отладки и тестирования алгоритмов, описанных в теоретических разделах настоящей работы. Предоставляется интуитивно понятный графический интерфейс с возможностью гибкой настройки параметров моделей, наборов данных и условий обучения. Благодаря использованию визуальных компонентов пользователь может в интерактивном режиме наблюдать за процессом формирования разделяющих поверхностей, анализировать выходы моделей, а также проводить тестирование устойчивости классификаторов.

Разработка велась с учётом необходимости масштабируемости архитектуры: структура системы разделена на независимые функциональные блоки, что обеспечивает возможность расширения и модификации без необходимости переписывания всего кода. Интерфейс системы логически организован по вкладкам, каждая из которых отвечает за определённую группу задач: генерация и загрузка данных, обучение модели, проведение экспериментов, визуализация и анализ результатов.

На момент завершения работы интеллектуальная система машинного обучения включает в себя следующие ключевые функциональные возможности:
\begin{itemize}
    \item настройка параметров архитектуры многослойного персептрона, включая размеры и количество слоёв, выбор функции активации, установку порогов доверия;
    \item управление параметрами обучения (функция потерь, оптимизатор, регуляризация, и т.д.);
    \item пошаговая визуализация процесса обучения, включая изменение выходов модели, метрик и формирование ячеек;
    \item реализация как классических методов бинарной классификации, так и модифицированного метода с фоном;
    \item визуализация, построение и загрузка обучающих и тестовых множеств;
    \item проведение экспериментальных исследований по созданию синтетических данных, обработке данных с пропусками, а также анализ объясняющего двоичного дерева eXBTree.
\end{itemize}

Таким образом, система реализует весь цикл исследования: от генерации обучающего множества до визуализации результатов и анализа поведения модели в различных условиях. Её применение позволяет не только демонстрировать основные методы, описанные в главах 1–3, но и проводить дополнительный количественный и качественный анализ, направленный на верификацию теоретических положений.
