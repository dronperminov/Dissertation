\section{Роль интеллектуальной системы в исследовании}

Разработанная интеллектуальная система машинного обучения стала ключевым инструментом, обеспечивающим как реализацию предложенных в диссертации методов, так и всестороннюю экспериментальную проверку их свойств. Её архитектура ориентирована на гибкую настройку параметров моделей, визуальный контроль над процессом обучения, а также глубокий анализ поведения классификатора в условиях, приближённых к реальным сценариям применения.

Интеллектуальная система позволила оперативно проверять влияние архитектурных параметров нейронной сети на её аппроксимационные способности. Использование многослойного персептрона в качестве доверенного аппроксиматора вероятности принадлежности объекта к классу потребовало детальной настройки числа слоёв, функции активации и порога отказа \(\beta\). Все эти параметры доступны для интерактивного изменения в ходе экспериментов.

Интеллектуальная система была использована для воспроизведения поведения модели на данных с перекрытием классов и в условиях неопределённости. Визуализация границ принятия решений и отказа от классификации показала, как повышение порога \(\beta\) повышает надёжность предсказаний за счёт исключения сомнительных точек. Такие наблюдения трудно формализовать численно, но они критичны для практической интерпретации поведения модели.

Кроме того, интеллектуальная система оказалась удобной платформой для отладки и тестирования предложенной модификации классификатора в условиях ограниченного объёма данных и несбалансированности классов. Возможность быстрой генерации обучающих выборок и отображения результатов классификации в реальном времени позволила провести сотни запусков, лежащих в основе статистической оценки качества.

Таким образом, интеллектуальная система машинного обучения выполнила не только вспомогательную, но и методологически значимую роль, обеспечив воспроизводимость, наглядность и полноту исследования. Её использование позволило обосновать теоретические положения диссертации эмпирически, за счёт детального анализа поведения моделей на управляемых синтетических данных.
