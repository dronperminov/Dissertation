\section{Структура и функциональные компоненты пользовательского интерфейса}

Разработанная интеллектуальная система машинного обучения имеет модульную архитектуру пользовательского интерфейса, организованного в виде тематических вкладок. Такой подход позволяет изолировать различные этапы исследования и обеспечить пользователю интуитивно понятную навигацию между функциональными блоками. Вкладки интерфейса включают: ``Данные``, ``Обучение`` и ``Эксперименты``. Каждая из вкладок реализует отдельный аспект взаимодействия с системой.

\paragraph{Вкладка ``Данные``.} На данной вкладке пользователь может создавать обучающие и тестовые выборки, управляя их параметрами с помощью наглядных графических контроллеров, а также загружать данные из csv файлов. В числе доступных настроек:
\begin{itemize}
    \item выбор числа объектов и размерности признаков;
    \item задание доли тестовых данных;
    \item задание доли объектов каждого класса;
    \item включение ошибок разметки;
    \item выполнение нормализации и стандартизации признаков.
\end{itemize}

Созданные или загруженные данные визуализируются на двумерной плоскости с использованием цветовой кодировки классов.

\paragraph{Вкладка ``Обучение``.} Данный раздел интерфейса предназначен для настройки архитектуры модели, параметров её обучения и визуального отображения различных элементов. Пользователь может:
\begin{itemize}
    \item выбирать структуру многослойного персептрона (число слоёв, нейронов, функций активации);
    \item задавать параметры оптимизации (тип оптимизатора, скорость обучения, параметры регуляризации, функцию потерь);
    \item управлять отображение данных, сетки, ячеек и режимом выхода модели;
    \item запускать процесс обучения с возможностью пошагового анализа и прерывания.
\end{itemize}

Обучение сопровождается в реальном времени визуализацией различных характеристик: поверхности выходной функции модели, изменения функции ошибки, доли ошибок на обучающем и тестовом множествах, а также показателей отказа от классификации в зависимости от порога доверия.

\paragraph{Вкладка ``Эксперименты``.} Этот раздел агрегирует инструменты для проведения углублённого анализа результатов. В частности, доступны:
\begin{itemize}
    \item визуализация объясняющего двоичного дерева;
    \item создание синтетических данных на основе обученной модели;
    \item обучение модели на данных с искусственно внесёнными пропусками.
\end{itemize}

Для каждой из опций предусмотрено графическое отображение и интуитивно понятная система управления с наглядными пояснениями каждого шага и полученных результатов. Это делает модуль эффективным инструментом для анализа моделей в условиях ограниченного количества данных, дисбаланса классов и наличия пропусков.
