\section{Алгоритмы визуализации}

Графическая составляющая интеллектуальной системы реализована с использованием низкоуровневых возможностей браузера, таких как SVG и HTML5 Canvas API. Отказ от сторонних библиотек в пользу чистого JavaScript и CSS обусловлен требованиями к производительности, контролю над прорисовкой и необходимости точной синхронизации между визуальными компонентами.

Отрисовка данных, поверхности выхода модели, границ принятия решений, зон отказа от классификации, а также метрик качества осуществляется в режиме реального времени. Обновление изображений происходит по мере поступления новых данных или изменения параметров модели.

Основные принципы реализации визуализации включают:

\begin{itemize}
    \item использование Canvas API для эффективной отрисовки цветных карт выхода модели;
    \item применение SVG для отрисовки больших массивов точек, осей, подписей, интерактивных маркеров и других элементов управления;
    \item разделение визуальных компонентов на независимые модули, каждый из которых регистрирует себя как подписчик событий модели и наборов данных;
    \item организация обмена сообщениями между компонентами через реализацию событийной модели на базе шаблона EventEmitter;
    \item использование аппаратного ускорения браузера при отрисовке и обновлении графических элементов;
    \item минимизация количества полных перерисовок за счёт дифференциального обновления слоёв.
\end{itemize}

Особое внимание уделяется синхронности всех отображаемых компонентов и их согласованности с текущим состоянием модели. Каждый визуализируемый объект автоматически обновляется при изменении состояния, что позволяет пользователю в реальном времени отслеживать последствия своих действий.
