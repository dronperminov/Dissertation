\section{Архитектура и интерфейс интеллектуальной системы}

Разработанная система реализована на JavaScript с использованием стандартных веб-технологий: HTML5~\cite{hickson2011html5}, CSS3~\cite{lunn2012css3}, SVG~\cite{quint2003scalable} и Canvas API~\cite{lubbers2010using}. Для стилизации применяется собственный CSS без привлечения сторонних фреймворков. Взаимодействие между компонентами построено на событийной модели с использованием собственного класса-эмиттера событий (EventEmitter). Основным управляющим объектом является класс Playground, который инкапсулирует логику координации работы всех компонентов и обмена данными между ними. Все вычисления производятся на стороне клиента, что исключает необходимость обращения к внешним серверам и обеспечивает полную автономность работы.

%%%%%%%%%%%%%%%%%%%%%%%%%%%%%%%%%%%%%%%%%%%%%%%%%%%%%%%%%%%%%%%%%%%%%%%%%%%%%%%%%%%%%%%%%%%%%%%%%%%%%%%%%%%%%%%
\subsection{Архитектура интеллектуальной системы}

Архитектура системы выполнена по модульному принципу, включая следующие ключевые блоки:

\begin{itemize}
    \item \textbf{Модуль данных} -- отвечает за генерацию, хранение и загрузку обучающих и тестовых наборов.
    \item \textbf{Модуль модели} -- реализует обучение персептрона и его использование для анализа.
    \item \textbf{Модуль обучения} -- реализует методы градиентного спуска, алгоритмы оптимизации и функций потерь, а также формирование модифицированных обучающих выборок с фоновым распределением.
    \item \textbf{Модуль визуализации} -- занимается отрисовкой данных, иерархии ячеек, карты предсказаний, структурных элементов модели, а также графиков метрик и гистограмм. Визуализация осуществляется с помощью Canvas API и SVG.
    \item \textbf{Модуль экспериментов} -- обеспечивает выполнение преднастроенных экспериментов, таких как анализ дерева eXBTree, создание синтетических данных и обучение модели на данных с пропусками.
    \item \textbf{Модуль управления} -- реализует пользовательский интерфейс, включая меню, формы настройки параметров и кнопки управления, а также обработку событий от пользователя.
\end{itemize}

Все модули взаимодействуют между собой через механизм событий, что обеспечивает слабую связанность и гибкость расширения. Класс Playground выступает центральным контроллером, инициализирующим компоненты, регистрирующим слушатели событий и передающим данные между модулями.

%%%%%%%%%%%%%%%%%%%%%%%%%%%%%%%%%%%%%%%%%%%%%%%%%%%%%%%%%%%%%%%%%%%%%%%%%%%%%%%%%%%%%%%%%%%%%%%%%%%%%%%%%%%%%%%
\subsection{Структура интерфейса}

Интерфейс интеллектуальной системы структурирован по трём основным вкладкам, каждая из которых реализована как независимый набор компонентов с собственным меню и областью отображения:

\begin{itemize}
    \item \textbf{Вкладка ``Данные``} -- предоставляет средства генерации и файловой загрузки наборов данных. Отображение данных осуществляется в табличном виде и на графике с цветовой кодировкой для обучающего и тестового разбиений. Имеются инструменты нормализации и экспорта данных.

    \item \textbf{Вкладка ``Обучение``} -- главный визуально насыщенный раздел, где происходит настройка архитектуры модели (число слоёв, размер слоёв, функции активации, порог доверия), параметров обучения (скорость, функция потерь, оптимизатор, регуляризация) и параметров визуализации (отображение выходов модели и формируемых ячеек, точки обучающего, тестового и фонового множеств). Обучение может как запускаться и останавливаться по желанию, так и выполняться в виде единственного шага. Область просмотра динамически отображает состояние модели, метрики и распределение выходов персептрона на обучающих данных в виде гистограмм.

    \item \textbf{Вкладка ``Эксперименты``} -- содержит инструменты для запуска и анализа различных сценариев: анализ двоичного объясняющего дерева, создание синтетических данных, а также обучение модели на данных с пропусками. Результаты представлены в виде интерактивных таблиц, графиков и гистограмм, позволяющих детально исследовать поведение модели.
\end{itemize}

Интерфейс спроектирован с акцентом на интерактивность и прозрачность процесса: изменение параметров мгновенно отражается на визуализации, что позволяет пользователю оперативно оценивать влияние настроек.

%%%%%%%%%%%%%%%%%%%%%%%%%%%%%%%%%%%%%%%%%%%%%%%%%%%%%%%%%%%%%%%%%%%%%%%%%%%%%%%%%%%%%%%%%%%%%%%%%%%%%%%%%%%%%%%
\subsection{Аппаратные и программные требования}

Интеллектуальная система машинного обучения не требует установки дополнительных библиотек или серверной инфраструктуры. Для работы необходим любой современный браузер с поддержкой Javascript, HTML5 Canvas и SVG. Ресурсоёмкость минимальна, что позволяет запускать систему на большинстве персональных компьютеров без специальных требований.
