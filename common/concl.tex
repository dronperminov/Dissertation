%% Согласно ГОСТ Р 7.0.11-2011:
%% 5.3.3 В заключении диссертации излагают итоги выполненного исследования, рекомендации, перспективы дальнейшей разработки темы.
%% 9.2.3 В заключении автореферата диссертации излагают итоги данного исследования, рекомендации и перспективы дальнейшей разработки темы.

\begin{enumerate}
    \item Разработан и реализован метод построения классификатора, обеспечивающего состоятельную аппроксимацию апостериорных вероятностей при дисбалансе классов и некомплектности данных за счёт использования модифицированного байесовского классификатора на основе многослойного персептрона.
    \item Разработаны методы генерации синтетических табличных данных и обработки некомплектных табличных данных на основе предложенного метода построения классификатора.
    \item Проведено экспериментальное исследование разработанных методов на модельных и прикладных данных для оценки устойчивости классификатора к дисбалансу классов, неполноте данных и корректности обработки объектов вне носителя обучающего распределения.
    \item Разработана интеллектуальная система машинного обучения, реализующая предложенные методы и обеспечивающая решение задач классификации табличных данных в условиях дисбаланса классов, некомплектности данных и высокой неопределённости вне носителя распределения.
\end{enumerate}
