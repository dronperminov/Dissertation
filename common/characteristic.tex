
{\actuality} 

Одной из ключевых задач в машинном обучении и статистическом распознавании образов является построение классификаторов, обладающих устойчивостью при ограниченных объёмах обучающих данных, а также при наличии существенного дисбаланса классов~\cite{he2009learning, japkowicz2002class}. Такие условия характерны для множества прикладных сценариев, в том числе в анализе табличных данных невысокой размерности, применяемых в медицинской диагностике, промышленном контроле, финансовой аналитике и других областях, где один из классов либо отсутствует в выборке частично, либо представлен незначительным числом наблюдений. При этом априорные вероятности классов могут значительно различаться, что приводит к деградации точности решений, принимаемых в пользу маломощных распределений. Невозможность обоснованно классифицировать объекты, относящиеся к слабо представленным кластерам, особенно при их близости к границам доминирующего класса, делает задачу построения классификатора в такой постановке математически и вычислительно нетривиальной~\cite{elkan2001foundations}.

Отдельную сложность представляет ситуация, когда тестовые данные выходят за пределы носителя распределения обучающей выборки. В таких случаях отсутствует статистически обоснованная возможность принимать решения с высокой степенью уверенности, и требуется формализованный механизм отказа от классификации. Это особенно важно в задачах, где цена ошибки велика, а распределения данных обладают сложной или разреженной структурой.

Известно, что при высокой априорной вероятности одного из классов даже значительное значение правдоподобия маломощного класса может оказаться недостаточным для принятия в его пользу в рамках классического байесовского подхода~\cite{bishop2006pattern}. В то же время стандартные процедуры балансировки, такие как редукция выборки доминирующего класса либо синтетическое увеличение мощности маломощных классов (например, путём дублирования наблюдений), как правило, вносят искажения в структуру исходного распределения~\cite{chawla2002smote}. Это нарушает статистические предпосылки, лежащие в основе байесовского вывода, и делает полученные модели плохо интерпретируемыми и неустойчивыми при генерализации.

Дополнительным фактором, затрудняющим построение устойчивых моделей, является неполнота данных. В табличных наборах, собираемых в реальных прикладных задачах, часто присутствуют пропуски, возникающие по разным причинам -- от случайных ошибок измерений до систематического отсутствия признаков у целых подмножеств объектов. Обработка таких данных требует специальных методов, обеспечивающих корректность статистического вывода и минимизацию потерь информации~\cite{little1995statistical}.

Актуальной задачей является построение классификатора, сохраняющего свойства состоятельности и обобщающей способности в условиях ограниченности данных, дисбаланса классов, неполноты наблюдений и возможного выхода входных данных за пределы области, охваченной обучающей выборкой. Дополнительную сложность представляет необходимость получения интерпретируемых и вычислительно эффективных правил принятия решений, особенно при высокой размерности пространства признаков и ограниченной мощности выборки~\cite{rudin2019stop}. Для практического применения классификатора важной становится также возможность локализации областей пространства признаков, в которых принимаемые решения обладают высокой степенью надёжности.

Особую актуальность приобретает задача получения состоятельной оценки апостериорной вероятности принадлежности к классу, заданной на компакте в пространстве признаков, без дополнительных предположений о параметрической форме плотностей~\cite{devroye2013probabilistic}. В условиях отсутствия полной априорной информации, ограниченного количества обучающих примеров, неполноты данных и неоднородности структуры распределения необходимы методы, обладающие адаптивностью, устойчивостью к локальным вариациям плотности, масштабной инвариантностью и возможностью отказа от классификации в зонах высокой неопределённости. Такие методы должны быть совместимы с современными требованиями вычислительной эффективности, масштабируемости и интерпретируемости~\cite{dwivedi2023explainable}.

Задача построения универсальных, статистически обоснованных и практически применимых моделей классификации в подобных условиях остаётся открытой и представляет собой предмет активного теоретического и прикладного исследования.



% Обзор, введение в тему, обозначение места данной работы в
% мировых исследованиях и~т.\:п., можно использовать ссылки на~другие
% работы~\autocite{Gosele1999161,Lermontov}
% (если их~нет, то~в~автореферате
% автоматически пропадёт раздел <<Список литературы>>). Внимание! Ссылки
% на~другие работы в~разделе общей характеристики работы можно
% использовать только при использовании \verb!biblatex! (из-за технических
% ограничений \verb!bibtex8!. Это связано с тем, что одна
% и~та~же~характеристика используются и~в~тексте диссертации, и в
% автореферате. В~последнем, согласно ГОСТ, должен присутствовать список
% работ автора по~теме диссертации, а~\verb!bibtex8! не~умеет выводить в~одном
% файле два списка литературы).
% При использовании \verb!biblatex! возможно использование исключительно
% в~автореферате подстрочных ссылок
% для других работ командой \verb!\autocite!, а~также цитирование
% собственных работ командой \verb!\cite!. Для этого в~файле
% \verb!common/setup.tex! необходимо присвоить положительное значение
% счётчику \verb!\setcounter{usefootcite}{1}!.

% Для генерации содержимого титульного листа автореферата, диссертации
% и~презентации используются данные из файла \verb!common/data.tex!. Если,
% например, вы меняете название диссертации, то оно автоматически
% появится в~итоговых файлах после очередного запуска \LaTeX. Согласно
% ГОСТ 7.0.11-2011 <<5.1.1 Титульный лист является первой страницей
% диссертации, служит источником информации, необходимой для обработки и
% поиска документа>>. Наличие логотипа организации на~титульном листе
% упрощает обработку и~поиск, для этого разметите логотип вашей
% организации в папке images в~формате PDF (лучше найти его в векторном
% варианте, чтобы он хорошо смотрелся при печати) под именем
% \verb!logo.pdf!. Настроить размер изображения с логотипом можно
% в~соответствующих местах файлов \verb!title.tex!  отдельно для
% диссертации и автореферата. Если вам логотип не~нужен, то просто
% удалите файл с~логотипом.

% \ifsynopsis
% Этот абзац появляется только в~автореферате.
% Для формирования блоков, которые будут обрабатываться только в~автореферате,
% заведена проверка условия \verb!\!\verb!ifsynopsis!.
% Значение условия задаётся в~основном файле документа (\verb!synopsis.tex! для
% автореферата).
% \else
% Этот абзац появляется только в~диссертации.
% Через проверку условия \verb!\!\verb!ifsynopsis!, задаваемого в~основном файле
% документа (\verb!dissertation.tex! для диссертации), можно сделать новую
% команду, обеспечивающую появление цитаты в~диссертации, но~не~в~автореферате.
% \fi

% {\progress}
% Этот раздел должен быть отдельным структурным элементом по
% ГОСТ, но он, как правило, включается в описание актуальности
% темы. Нужен он отдельным структурынм элемементом или нет ---
% смотрите другие диссертации вашего совета, скорее всего не нужен.

{\aim} данной работы является построение формально обоснованных методов классификации табличных данных, обеспечивающих статистическую состоятельность, устойчивость к дисбалансу классов и некомплектности данных, а также корректную обработку объектов вне носителя обучающего распределения, на основе модифицированного байесовского классификатора с использованием многослойного персептрона.

Для~достижения поставленной цели необходимо было решить следующие {\tasks}:
\begin{enumerate}[beginpenalty=10000] % https://tex.stackexchange.com/a/476052/104425
  \item Разработать и реализовать метод построения классификатора, обеспечивающего состоятельную аппроксимацию апостериорных вероятностей при дисбалансе классов и некомплектности данных за счёт использования модифицированного байесовского классификатора на основе многослойного персептрона.

  \item Разработать методы генерации синтетических табличных данных и обработки некомплектных табличных данных на основе предложенного метода построения классификатора.

  \item Провести экспериментальное исследование разработанных методов на модельных и прикладных данных для оценки устойчивости классификатора к дисбалансу классов, неполноте данных и корректности обработки объектов вне носителя обучающего распределения.

  \item Разработать интеллектуальную систему машинного обучения, реализующую предложенные методы и обеспечивающую решение задач классификации табличных данных в условиях дисбаланса классов, некомплектности данных и высокой неопределённости вне носителя распределения.
\end{enumerate}

{\defpositions}
\begin{enumerate}[beginpenalty=10000] % https://tex.stackexchange.com/a/476052/104425
  \item Метод построения классификатора, обеспечивающего состоятельную аппроксимацию апостериорных вероятностей при дисбалансе классов и некомплектности данных за счёт использования модифицированного байесовского классификатора на основе многослойного персептрона.
  
  \item Метод создания синтетических табличных данных на основе предложенного метода построения классификатора.

  \item Метод обработки некомплектных табличных данных, на основе предложенного метода построения классификатора.

  \item Интеллектуальная система машинного обучения, реализующая предложенные методы и обеспечивающая решение задач классификации табличных данных в условиях дисбаланса классов, некомплектности данных и высокой неопределённости вне носителя распределения.
\end{enumerate}

Перечисленные положения относятся к направлениям исследований 4, 7, 8 и 9 паспорта специальности 2.3.5.

{\novelty} разработан метод построения классификаторов табличных данных, обеспечивающий состоятельную аппроксимацию апостериорных вероятностей в условиях дисбаланса классов, неполноты данных и выхода объектов за пределы носителя обучающего распределения. Метод основан на аппроксимации апостериорных вероятностей с использованием многослойного персептрона с кусочно-линейными функциями активации и включает две специализированные версии: бинарную для аппроксимации вероятности между двумя классами и унарную для оценки плотности распределения одного класса с формализованной процедурой отказа от классификации и детектированием объектов вне обучающего распределения. В рамках метода предложены алгоритмы обработки данных с пропусками на основе унарной вероятностной модели и генерации синтетических табличных данных с сохранением статистических характеристик исходного распределения. Дополнительно выполнено развитие теоретических основ построения классификаторов в условиях ограниченного объёма данных и высокой неопределённости, включая формализацию процедур отказа и введение количественных показателей доверия к решениям.

Основные элементы научной новизны состоят в следующем:

\begin{enumerate}[beginpenalty=10000] % https://tex.stackexchange.com/a/476052/104425
  \item Установлена асимптотическая эквивалентность между нейросетевой регрессией, построенной с использованием многослойного персептрона, и гистограммной оценкой апостериорной вероятности, что обеспечивает теоретическое обоснование состоятельности предложенного метода.
  \item Введён формализованный механизм отказа от классификации, основанный на пороговой оценке аппроксимированной апостериорной вероятности, и предложен способ количественного измерения уровня доверия к решению классификатора в каждом наблюдении признакового пространства.
  \item Разработан метод обучения по некомплектным данным, основанный на совместном восстановлении пропущенных значений и обучении унарных регрессионных моделей, что обеспечивает устойчивость к пропускам и позволяет проводить классификацию с использованием частичной информации.
  \item Предложен метод создания синтетических табличных данных путём прореживания равномерного фонового распределения с использованием адаптивной плотностной аппроксимации, полученной в результате унарной классификации, с доказанной состоятельностью соответствующих оценок.
  \item Показано, что предложенный классификатор обладает линейной сложностью по числу объектов выборки при применении, что обеспечивает его пригодность для обработки больших табличных наборов данных в реальном времени.
\end{enumerate}

{\influence}

Теоретическая значимость работы состоит в развитии математических основ байесовской классификации в условиях дисбаланса классов, ограниченного объёма выборки, отсутствующих данных и выхода за пределы носителя обучающего распределения. Установлена асимптотическая связь между нейросетевой и гистограммной оценками апостериорной вероятности, обеспечивающая обоснование состоятельности предложенного подхода. Введён формализованный критерий отказа от классификации, основанный на пороговой аппроксимации апостериорной вероятности, и предложена процедура количественной оценки уровня доверия к принимаемым классификатором решениям. Разработанный подход расширяет теоретическую базу классификации и дополняет существующие модели нелинейной регрессии в вероятностном контексте, обеспечивая строгие условия применимости и гарантии корректности работы в зонах высокой неопределённости.

Практическая значимость заключается в возможности применения разработанного метода к задачам анализа табличных данных в условиях ограниченной или неполной информации. Предложенный классификатор позволяет осуществлять устойчивую и интерпретируемую классификацию в задачах с несбалансированными классами и повышать надёжность принимаемых решений за счёт автоматического отказа от распознавания в недостоверных областях признакового пространства. Разработаны методы генерации синтетических данных, сохраняющих структуру исходного распределения, а также методы обработки пропущенных значений на основе регрессионных унарных моделей. Реализован программный комплекс, обеспечивающий воспроизводимое и наглядное применение предложенного метода в задачах предварительного анализа классификации и восстановления табличных данных, а также оценке уровня доверия обученных моделей.

{\probation}
Основные результаты работы были представлены на следующих конференциях и семинарах:

\begin{itemize}
    \item Форум «Цифровая экономика. Технологии доверенного искусственного интеллекта», Москва, 25 мая 2023 г.
    \item 32-я научно-техническая конференция «Методы и технические средства обеспечения безопасности информации» (МиТСОБИ), Санкт-Петербург, 26-29 июня 2023 г.
    \item WAIT: Workshop on Artificial Intelligence Trustworthiness, Almaty, Kazakhstan, 24 апреля 2024 г.
    \item Международная конференция «Иванниковские чтения», Великий Новгород, 17-18 мая 2024 г.
    \item II форум «Технологии доверенного искусственного интеллекта», Москва, 27 мая 2024 г.    
    \item 33-я научно-техническая конференция «Методы и технические средства обеспечения безопасности информации» (МиТСОБИ), Санкт-Петербург, 24-27 июня 2024 г.
    \item MathAI 2025 The International Conference dedicated to mathematics in artificial intelligence, March 24-28, 2025 г.
    \item III форум «Технологии Доверенного Искусственного Интеллекта», Москва, 20 мая 2025 г.
    \item 34-я всероссийская конференция «Методы и технические средства обеспечения безопасности информации» (МиТСОБИ), Санкт-Петербург, 23-26 июня 2025 г.
    \item Международная конференция «Иванниковские чтения», Иркутск, 26-27 июня 2025 г.
\end{itemize}

{\contribution} Все выносимые на защиту результаты получены лично автором.

\ifnumequal{\value{bibliosel}}{0}
{%%% Встроенная реализация с загрузкой файла через движок bibtex8. (При желании, внутри можно использовать обычные ссылки, наподобие `\cite{vakbib1,vakbib2}`).
    {\publications} Основные результаты по теме диссертации изложены
    в~XX~печатных изданиях,
    X из которых изданы в журналах, рекомендованных ВАК,
    X "--- в тезисах докладов.
}%
{%%% Реализация пакетом biblatex через движок biber
    \begin{refsection}[bl-author, bl-registered]
        % Это refsection=1.
        % Процитированные здесь работы:
        %  * подсчитываются, для автоматического составления фразы "Основные результаты ..."
        %  * попадают в авторскую библиографию, при usefootcite==0 и стиле `\insertbiblioauthor` или `\insertbiblioauthorgrouped`
        %  * нумеруются там в зависимости от порядка команд `\printbibliography` в этом разделе.
        %  * при использовании `\insertbiblioauthorgrouped`, порядок команд `\printbibliography` в нём должен быть тем же (см. biblio/biblatex.tex)
        %
        % Невидимый библиографический список для подсчёта количества публикаций:
        \printbibliography[heading=nobibheading, section=1, env=countauthorvak,          keyword=biblioauthorvak]%
        \printbibliography[heading=nobibheading, section=1, env=countauthorwos,          keyword=biblioauthorwos]%
        \printbibliography[heading=nobibheading, section=1, env=countauthorscopus,       keyword=biblioauthorscopus]%
        \printbibliography[heading=nobibheading, section=1, env=countauthorconf,         keyword=biblioauthorconf]%
        \printbibliography[heading=nobibheading, section=1, env=countauthorother,        keyword=biblioauthorother]%
        \printbibliography[heading=nobibheading, section=1, env=countregistered,         keyword=biblioregistered]%
        \printbibliography[heading=nobibheading, section=1, env=countauthorpatent,       keyword=biblioauthorpatent]%
        \printbibliography[heading=nobibheading, section=1, env=countauthorprogram,      keyword=biblioauthorprogram]%
        \printbibliography[heading=nobibheading, section=1, env=countauthor,             keyword=biblioauthor]%
        \printbibliography[heading=nobibheading, section=1, env=countauthorvakscopuswos, filter=vakscopuswos]%
        \printbibliography[heading=nobibheading, section=1, env=countauthorscopuswos,    filter=scopuswos]%
        %
        \nocite{*}%
        %
        {\publications} Основные результаты по теме диссертации изложены в~\arabic{citeauthor}~печатных изданиях,
        \arabic{citeauthorvak} из которых изданы в журналах, рекомендованных ВАК\sloppy%
        \ifnum \value{citeauthorscopuswos}>0%
            , \arabic{citeauthorscopuswos} "--- в~периодических научных журналах, индексируемых Web of~Science и Scopus\sloppy%
        \fi%
        \ifnum \value{citeauthorconf}>0%
            , \arabic{citeauthorconf} "--- в~тезисах докладов.
        \else%
            .
        \fi%
        \ifnum \value{citeregistered}=1%
            \ifnum \value{citeauthorpatent}=1%
                Зарегистрирован \arabic{citeauthorpatent} патент.
            \fi%
            \ifnum \value{citeauthorprogram}=1%
                Зарегистрирована \arabic{citeauthorprogram} программа для ЭВМ.
            \fi%
        \fi%
        \ifnum \value{citeregistered}>1%
            Зарегистрированы\ %
            \ifnum \value{citeauthorpatent}>0%
            \formbytotal{citeauthorpatent}{патент}{}{а}{}\sloppy%
            \ifnum \value{citeauthorprogram}=0 . \else \ и~\fi%
            \fi%
            \ifnum \value{citeauthorprogram}>0%
            \formbytotal{citeauthorprogram}{программ}{а}{ы}{} для ЭВМ.
            \fi%
        \fi%
        % К публикациям, в которых излагаются основные научные результаты диссертации на соискание учёной
        % степени, в рецензируемых изданиях приравниваются патенты на изобретения, патенты (свидетельства) на
        % полезную модель, патенты на промышленный образец, патенты на селекционные достижения, свидетельства
        % на программу для электронных вычислительных машин, базу данных, топологию интегральных микросхем,
        % зарегистрированные в установленном порядке.(в ред. Постановления Правительства РФ от 21.04.2016 N 335)
    \end{refsection}%
    \begin{refsection}[bl-author, bl-registered]
        % Это refsection=2.
        % Процитированные здесь работы:
        %  * попадают в авторскую библиографию, при usefootcite==0 и стиле `\insertbiblioauthorimportant`.
        %  * ни на что не влияют в противном случае
        \nocite{vakbib2}%vak
        \nocite{patbib1}%patent
        \nocite{progbib1}%program
        \nocite{bib1}%other
        \nocite{confbib1}%conf
    \end{refsection}%
        %
        % Всё, что вне этих двух refsection, это refsection=0,
        %  * для диссертации - это нормальные ссылки, попадающие в обычную библиографию
        %  * для автореферата:
        %     * при usefootcite==0, ссылка корректно сработает только для источника из `external.bib`. Для своих работ --- напечатает "[0]" (и даже Warning не вылезет).
        %     * при usefootcite==1, ссылка сработает нормально. В авторской библиографии будут только процитированные в refsection=0 работы.
}

% При использовании пакета \verb!biblatex! будут подсчитаны все работы, добавленные
% в файл \verb!biblio/author.bib!. Для правильного подсчёта работ в~различных
% системах цитирования требуется использовать поля:
% \begin{itemize}
%         \item \texttt{authorvak} если публикация индексирована ВАК,
%         \item \texttt{authorscopus} если публикация индексирована Scopus,
%         \item \texttt{authorwos} если публикация индексирована Web of Science,
%         \item \texttt{authorconf} для докладов конференций,
%         \item \texttt{authorpatent} для патентов,
%         \item \texttt{authorprogram} для зарегистрированных программ для ЭВМ,
%         \item \texttt{authorother} для других публикаций.
% \end{itemize}
% Для подсчёта используются счётчики:
% \begin{itemize}
%         \item \texttt{citeauthorvak} для работ, индексируемых ВАК,
%         \item \texttt{citeauthorscopus} для работ, индексируемых Scopus,
%         \item \texttt{citeauthorwos} для работ, индексируемых Web of Science,
%         \item \texttt{citeauthorvakscopuswos} для работ, индексируемых одной из трёх баз,
%         \item \texttt{citeauthorscopuswos} для работ, индексируемых Scopus или Web of~Science,
%         \item \texttt{citeauthorconf} для докладов на конференциях,
%         \item \texttt{citeauthorother} для остальных работ,
%         \item \texttt{citeauthorpatent} для патентов,
%         \item \texttt{citeauthorprogram} для зарегистрированных программ для ЭВМ,
%         \item \texttt{citeauthor} для суммарного количества работ.
% \end{itemize}
% % Счётчик \texttt{citeexternal} используется для подсчёта процитированных публикаций;
% % \texttt{citeregistered} "--- для подсчёта суммарного количества патентов и программ для ЭВМ.

% Для добавления в список публикаций автора работ, которые не были процитированы в
% автореферате, требуется их~перечислить с использованием команды \verb!\nocite! в
% \verb!Synopsis/content.tex!.
